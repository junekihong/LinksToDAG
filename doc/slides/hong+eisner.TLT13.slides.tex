\documentclass{beamer}

% For more theme options:
%/usr/share/texmf/tex/latex/beamer/base/themes/theme/compatibility
\usepackage{beamerthemelined}

% this package seems to throw an error for me. -Juneki 12/6/14
%\usepackage[usenames,dvipsnames,svgnames,table]{xcolor}
\usepackage{soul}

\usepackage{algorithm}
\usepackage[noend]{algpseudocode}
\usepackage{graphicx}
\usepackage{caption}
\usepackage{subcaption}

\usepackage{tikz-dependency}

\newcommand{\eqnref}[1]{\eqref{eqn:#1}}
%\usepackage[usenames,dvipsnames,svgnames,table]{xcolor}  % allows better color names
\usepackage{todonotes}   % insert [disable] to disable all notes
\newcommand{\Note}[4][]{\todo[author=#2,color=#3,fancyline,#1]{#4}}
\newcommand{\noteJH}[2][]{\Note[#1]{JH}{blue!40}{#2}}
\newcommand{\noteJE}[2][]{\Note[#1]{JE}{green!40}{#2}}
\newcommand{\notewho}[3][]{\Note[#1]{#2}{orange!40}{#3}}  % extra arg with miscellaneous author
\newcommand{\NoteJH}[2][]{\noteJH[inline,#1]{#2}}
\newcommand{\NoteJE}[2][]{\noteJE[inline,#1]{#2}}
\newcommand{\Notewho}[3][]{\notewho[inline,#1]{#2}{#3}}  % extra arg with miscellaneous author


\setbeamerfont{page number in head/foot}{}
\setbeamertemplate{footline}[frame number]

\begin{document}
\title{Deriving Multi-Headed Planar Dependency Parses from Link Grammar Parses}
\author{Juneki Hong, Jason Eisner}
\date{\today}
\frame{\titlepage} %\NoteJH{Include pictures of us?}

\section{Motivation}

%\subsection{Multi-headed}
\frame{\frametitle{Multi-headedness}
  Multi-headedness Can Capture Additional Linguistic Phenomenon
  \begin{itemize}
    \item Control
    \item Relativization
    \item Conjunction
  \end{itemize}
}

\subsection{Control}
\frame{\frametitle{Control}
  \begin{figure}
    \begin{dependency}
      \begin{deptext}
        Jill \& likes \& to \& skip \\
      \end{deptext}
      \deproot[edge above, thick, hide label, edge unit distance = 1.5ex]{2}{}
      \depedge[edge above, thick, hide label]{2}{1}{}
      \depedge[edge below, ultra thick, hide label, edge style = {purple}, edge unit distance = 1.2ex]{4}{1}{}
      \depedge[edge above, thick, hide label]{2}{3}{}
      \depedge[edge above, thick, hide label]{3}{4}{}
    \end{dependency}
    \caption*{Jill is the subject of two verbs}
  \end{figure}

  \begin{figure}
    \begin{dependency}
      \begin{deptext}
        Jill \& persuaded \& Jack \& to \& skip \\
      \end{deptext}
      \deproot[edge above, thick, hide label, edge unit distance = 1.5ex]{2}{}
      \depedge[edge above, thick, hide label]{2}{1}{}
      \depedge[edge above, thick, hide label]{2}{3}{}
      \depedge[edge above, thick, hide label]{3}{4}{}
      \depedge[edge above, thick, hide label]{4}{5}{}
      \depedge[edge below, ultra thick, hide label, edge style = {purple}, edge unit distance = 1.5ex]{3}{5}{}
    \end{dependency}
    \caption*{Jack is the object of one verb and the subject of another}
  \end{figure}
}


\subsection{Relativization}
\frame{\frametitle{Relativization}
  \begin{figure}
    \begin{dependency}
      \begin{deptext}
        The \& boy \& that \& Jill \& skipped \& with \& fell \& down \\
      \end{deptext}
      \deproot[edge above, thick, hide label, edge unit distance = 2ex]{7}{}
      \depedge[edge above, thick, hide label]{2}{1}{}
      \depedge[edge above, thick, hide label]{2}{3}{}
      \depedge[edge above, thick, hide label]{3}{5}{}
      \depedge[edge above, thick, hide label]{5}{4}{}
      \depedge[edge above, thick, hide label]{5}{6}{}
      \depedge[edge above, thick, hide label, edge unit distance = 1.8ex]{7}{2}{}
      \depedge[edge above, thick, hide label]{7}{8}{}
      \depedge[edge below, ultra thick, hide label, edge style = {purple}, edge unit distance = 1.5ex]{6}{2}{}
    \end{dependency}
    \caption*{The boy is the object of \textit{with} as well as the subject of \textit{fell}.}
  \end{figure}
}


\subsection{Conjunction}
\frame{\frametitle{Conjunction}
  \begin{figure}
    \begin{dependency}
      \begin{deptext}
        Jack \& and \& Jill \& went \& up \& the \& hill \\
      \end{deptext}
      \deproot[edge above, thick, hide label, edge unit distance = 2ex]{4}{}
      \depedge[edge above, thick, hide label]{2}{1}{}
      \depedge[edge above, thick, hide label]{2}{3}{}
      \depedge[edge above, thick, hide label]{4}{2}{}
      \depedge[edge above, thick, hide label]{4}{5}{}
      \depedge[edge above, thick, hide label]{5}{7}{}
      \depedge[edge above, thick, hide label]{7}{6}{}

      \depedge[edge below, ultra thick, hide label, edge style = {purple}, edge unit distance = 1.5ex]{4}{1}{}
      \depedge[edge below, ultra thick, hide label, edge style = {purple}, edge unit distance = 1.5ex]{4}{3}{}

    \end{dependency}
    \caption*{Jack and Jill serve as the two arguments to \textit{and}, but are also subjects of \textit{went}.}
  \end{figure}
}



\subsection{Motivation}
\frame{\frametitle{Motivation}
  \begin{itemize}
    \uncover<1->{\item A multiheaded dependency corpus would be useful for testing new parsing algorithms}
    \uncover<2->{\item Such a corpus could be automatically annotated using Integer Linear Programming}
    \uncover<3->{\item We explored whether the Link Grammar could be adapted for this purpose.}
    \uncover<4->{\item The results of this are mixed, but provides a good case study.}
  \end{itemize}
}







% Shouldn't mention non-projectivity. 
%\frame{\frametitle{Non-projectivity}
%\begin{figure}
%  \includegraphics[width=\linewidth]{multiheaded}
%  \caption*{English is not a completely projective language}
%  \NoteJH{image taken from Chris Dyer's slide: http://demo.clab.cs.cmu.edu/fa2014-11711/images/0/0e/Depparsing.pdf}
%  \NoteJH{I should also ask Chris for permission to use this}
%\end{figure}
%}


%\section{Non Projectivity}
%\frame{\frametitle{We do not address: Non-Projectivity}
%\begin{itemize}
%  \item Projectivity assumptions allow us to have efficient parsing algorithms that utilize dynamic programming
%  \item English is mostly projective
%\end{itemize}

%\NoteJH{Should I mention non-projectivity? This slide is mostly here because of a reviewer comment.}
%}

%\frame{\frametitle{We do not address: Non-Projectivity}
%\begin{itemize}
%  \item Projectivity assumptions allow us to have efficient parsing algorithms that utilize dynamic programming
%  \item English is mostly projective
%\end{itemize}

%\begin{figure}
%  \includegraphics[width=0.7\linewidth]{multiheaded}
%  \caption*{But not always}
%  \NoteJH{image taken from Chris Dyer's slide: http://demo.clab.cs.cmu.edu/fa2014-11711/images/0/0e/Depparsing.pdf}
%  \NoteJH{I should also ask Chris for permission to use this}
%\end{figure}


%}


\section{Link Grammars}

\frame{\frametitle{Link Grammars}
  \begin{itemize}
    \item[] Grammar-based formalism for projective dependency parsing with undirected links
    %\item We set out to discover if the undirected links are actually directed relationships.
  \end{itemize}
}


\frame{\frametitle{Link Grammars: Example Parse}
\begin{figure}
  \begin{dependency}[edge style={-}]
	\begin{deptext}
	  - \& n-u \& v \& e \& e \& v \& v-d \& r \& n-u \& - \\
	  the \& matter \& may \& never \& even \& be \& tried \& in \& court \& . \\
	\end{deptext}
	\depedge[edge above, thick, edge style = {red}]{2}{3}{S}
	\depedge[edge above, thick, edge style = {red}]{3}{6}{I}
	\depedge[edge above, thick, edge style = {red}]{2}{1}{D}
	\depedge[edge above, thick, edge style = {red}]{7}{8}{MV}
	\depedge[edge above, thick, edge style = {red}]{6}{7}{P}
	\depedge[edge above, thick, edge style = {red}]{8}{9}{J}
	\deproot[edge above, thick, edge style = {red}]{2}{W}
	\deproot[edge above, thick, edge style = {red}]{7}{WV}
	\deproot[edge above, thick, edge style = {red}]{10}{X}
	\depedge[edge above, thick, edge style = {red}]{6}{4}{E}
	\depedge[edge above, thick, edge style = {red}]{6}{5}{E}
  \end{dependency}
  \caption*{Link Parse of a sentence from Penn Tree Bank}
\end{figure}


}

\frame{\frametitle{Link Grammars: Converting into a Directed Acyclic Graph}
\begin{figure}
  \begin{dependency}
	\begin{deptext}
	  - \& n-u \& v \& e \& e \& v \& v-d \& r \& n-u \& - \\
	  the \& matter \& may \& never \& even \& be \& tried \& in \& court \& . \\
	\end{deptext}
	\depedge[edge above, thick, edge style = {red}]{2}{3}{S}
	\depedge[edge above, thick, edge style = {red}]{3}{6}{I}
	\depedge[edge above, thick, edge style = {red}]{2}{1}{D}
	\depedge[edge above, thick, edge style = {red}]{7}{8}{MV}
	\depedge[edge above, thick, edge style = {red}]{6}{7}{P}
	\depedge[edge above, thick, edge style = {red}]{8}{9}{J}
	\deproot[edge above, thick, edge style = {red}]{2}{W}
	\deproot[edge above, thick, edge style = {red}]{7}{WV}
	\deproot[edge above, thick, edge style = {red}]{10}{X}
	\depedge[edge above, thick, edge style = {red}]{6}{4}{E}
	\depedge[edge above, thick, edge style = {red}]{6}{5}{E}
  \end{dependency}

  \caption*{Directionalize the edges}
\end{figure}
}


\frame{\frametitle{Link Grammars}
Compare resulting dependency parse with CoNLL 2007 shared task. 
\begin{figure}
  \begin{dependency}
	\begin{deptext}
	  - \& n-u \& v \& e \& e \& v \& v-d \& r \& n-u \& - \\
	  the \& matter \& may \& never \& even \& be \& tried \& in \& court \& . \\
	  {\scriptsize DT} \& {\scriptsize NN} \& {\scriptsize MD} \& {\scriptsize RB} \& {\scriptsize RB} \& {\scriptsize VB} \& {\scriptsize VB} \& {\scriptsize IN} \& {\scriptsize NN} \& {\scriptsize .} \\
	\end{deptext}
	\deproot[edge below, thick, edge style = {blue}]{3}{\small ROOT}
	\depedge[edge above, thick, edge style = {red}]{2}{3}{S}
	\depedge[edge below, thick, edge style = {blue}]{3}{2}{\small SBJ}
	\depedge[edge below, thick, edge style = {blue}]{3}{4}{\small ADV}
	\depedge[edge below, thick, edge style = {blue}]{3}{5}{\small ADV}
	\depedge[edge above, thick, edge style = {red}]{3}{6}{I}
	\depedge[edge below, thick, edge style = {blue}]{3}{6}{\small VC}
	\depedge[edge below, thick, edge style = {blue}, edge unit distance =1.5ex]{3}{10}{\small P}
	\depedge[edge above, thick, edge style = {red}]{2}{1}{D}
	\depedge[edge below, thick, edge style = {blue}]{2}{1}{\small NMOD}
	\depedge[edge above, thick, edge style = {red}]{7}{8}{MV}
	\depedge[edge below, thick, edge style = {blue}]{7}{8}{\small ADV}
	\depedge[edge above, thick, edge style = {red}]{6}{7}{P}
	\depedge[edge below, thick, edge style = {blue}]{6}{7}{\small VC}
	\depedge[edge above, thick, edge style = {red}]{8}{9}{J}
	\depedge[edge below, thick, edge style = {blue}]{8}{9}{\small PMOD}
	\deproot[edge above, thick, edge style = {red}]{2}{W}
	\deproot[edge above, thick, edge style = {red}]{7}{WV}
	\deproot[edge above, thick, edge style = {red}]{10}{X}
	\depedge[edge above, thick, edge style = {red}]{6}{4}{E}
	\depedge[edge above, thick, edge style = {red}]{6}{5}{E}
  \end{dependency}

  \caption*{Top half is CoNLL. Bottom half is the directionalized link parse.}
\end{figure}


}



\frame{\frametitle{Link Grammars}
Compare resulting dependency parse with CoNLL 2007 shared task. 
\begin{figure}
  \begin{dependency}
	\begin{deptext}
	  - \& n-u \& v \& e \& e \& v \& v-d \& r \& n-u \& - \\
	  the \& matter \& may \& never \& even \& be \& tried \& in \& court \& . \\
	  {\scriptsize DT} \& {\scriptsize NN} \& {\scriptsize MD} \& {\scriptsize RB} \& {\scriptsize RB} \& {\scriptsize VB} \& {\scriptsize VB} \& {\scriptsize IN} \& {\scriptsize NN} \& {\scriptsize .} \\
	\end{deptext}
	\deproot[edge below, edge style = {blue, dotted}]{3}{\small ROOT}
	\depedge[edge above, edge style = {red, ultra thick}]{2}{3}{S}
	\depedge[edge below, edge style = {blue, ultra thick}]{3}{2}{\small SBJ}
	\depedge[edge below, edge style = {blue, dotted}]{3}{4}{\small ADV}
	\depedge[edge below, edge style = {blue, dotted}]{3}{5}{\small ADV}
	\depedge[edge above, edge style = {red, thick}]{3}{6}{I}
	\depedge[edge below, edge style = {blue, thick}]{3}{6}{\small VC}
	\depedge[edge below, edge style = {blue, dotted}, edge unit distance =1.5ex]{3}{10}{\small P}
	\depedge[edge above, edge style = {red, thick}]{2}{1}{D}
	\depedge[edge below, edge style = {blue, thick}]{2}{1}{\small NMOD}
	\depedge[edge above, edge style = {red, thick}]{7}{8}{MV}
	\depedge[edge below, edge style = {blue, thick}]{7}{8}{\small ADV}
	\depedge[edge above, edge style = {orange, thick}]{6}{7}{P}
	\depedge[edge below, edge style = {blue, thick}]{6}{7}{\small VC}
	\depedge[edge above, edge style = {red, thick}]{8}{9}{J}
	\depedge[edge below, edge style = {blue, thick}]{8}{9}{\small PMOD}
	\deproot[edge above, edge style = {red, dotted}]{2}{W}
	\deproot[edge above, edge style = {orange, ultra thick, dotted}]{7}{WV}
	\deproot[edge above, edge style = {red, dotted}]{10}{X}
	\depedge[edge above, edge style = {red, dotted}]{6}{4}{E}
	\depedge[edge above, edge style = {red, dotted}]{6}{5}{E}
  \end{dependency}

  \caption*{Top half is CoNLL. Bottom half is the directionalized link parse.}
\end{figure}


}


%\section{ILP}
%\frame{\frametitle{Integer Linear Programming}

%\NoteJH{Will the audience know about ILP?}

%}


\section{ILP Model}


\frame{\frametitle{Integer Linear Programming Model}

Encoded Constraints:
\begin{itemize}
  \item Connectedness
  \item Acyclicity
  \item Consistency of Directionalized Links
\end{itemize}
}



\frame{\frametitle{Integer Linear Programming Model}

For each sentence, for each edge $i,j$, where $i < j$

\begin{figure}
  \begin{dependency}[edge style={-}]
    \begin{deptext}
      . \& . \& . \& $i$ \& . \& . \& . \& $j$ \& . \& . \& . \\
    \end{deptext}
    \depedge[edge above, thick, edge unit distance = 1.1ex]{4}{8}{L}
  \end{dependency}
\end{figure}



Variables: 
\begin{itemize}
\item[] $x_{ij}, x_{ji} \in \mathbb{Z} \geq 0$: orientation of each link
\item[] $x_{ij} + x_{ji} = 1$

%\item $x_{ij} + x_{ji} = 1$ (A link can only be oriented left or right)

\end{itemize}





%\NoteJH{TODO: Incomplete slide. Need to describe things like depth of the variable, label}
}



\frame{\frametitle{Connectedness, Acyclicity}
\begin{itemize}
  \item[] Connectedness
    \begin{align}
      \sum_u x_{uv} & \geq 1
    \end{align}

  \item[] Acyclicity
    \begin{itemize}
      \item[] Given that node $u$ is the parent of $v$
      \item[] $n_v$: length of the sentence containing node $v$
      \item[] $d_v \in [0, n_v]$: depth of the node from the root of the sentence
    \end{itemize}

    \begin{align}
      (\forall_u)\; d_v + (1 + n_v) \cdot (1 - x_{uv}) & \geq 1+d_u
    \end{align}
\end{itemize}

}


\frame{\frametitle{Consistency of Directionalized Links \onslide<2->{with Slack}}
\begin{itemize}
  \item[] Consistency of Directionalized Links
    \begin{itemize}
      \item[] $r_L, \ell_L \in \{0,1\}$: whether links with label $L$ allowed left/right
    \end{itemize}
    
    \begin{align}\label{direction+slack}
      x_{ij} &\leq r_L \onslide<2->{ + s_{ij}} &
      x_{ji} &\leq \ell_L \onslide<2->{ + s_{ij}}
    \end{align}

    \begin{align}\label{eqn:obj}
      \min \left( \sum_L r_L + \ell_L \right) \onslide<2->{\cdot \frac{N_L}{4} + \sum_{ij}s_{ij}}
    \end{align}

    \onslide<2->{\begin{itemize}
      \item[] $s_{ij} \in \mathbb{R} \geq 0$: slack variable 
      \item[] $N_L$: Number of link tokens with label $L$
    \end{itemize}

    \begin{itemize}
      \item[] Slack allows a few links with label $L$ in disallowed directions
    \end{itemize}
    }

\end{itemize}
}


\section{Experiments and Results}
%\frame{\frametitle{Experiments and Results}
%\begin{itemize}
%  \item 18,577 English sentences with gold CoNLL. 
%  \item 18,577 unlabeled Russian sentences.
%\end{itemize}
%}

\frame{\frametitle{Data Sets}
%\begin{itemize}
%  \item[] 18,577 English sentences with 10,960 connected parses
%  \item[] 18,577 unlabeled Russian sentences with 4,913 connected parses
%\end{itemize}
  \begin{table}[h]
    \begin{tabular}{|l|l|l|}
      \hline
      & \# Sentences & \# Connected Parses \\ \hline
      English             & 18,577        & 10,960               \\ \hline
      Russian (unlabeled) & 18,577        & 4,913                \\ \hline
    \end{tabular}
  \end{table}
  
  On the English data set:
  
  \begin{itemize}
  \item[] Link Data has 8\% additional edges over the CoNLL.
  \item[] 52\% of links match CoNLL arcs
  \item[] 57\% of CoNLL arcs match links
  \end{itemize}
  
  \NoteJH{I skipped the ``Stability of Results'' section}
}

%\frame{\frametitle{On the English data set}
%\begin{itemize}
%\item[] Link Data has 8\% additional edges over the CoNLL.
%\item[] 52\% of links match CoNLL arcs
%\item[] 57\% of CoNLL arcs match links
%\end{itemize}

%
%}

\frame{\frametitle{ILP Results}
%  \begin{table}[h]
%    \begin{tabular}{|l|l||l|}
%      \hline
%      & & out of total \\ \hline
%      Link types & & \\allowing both directions & 7 & 113 \\ \hline
%      Link tokens & & \\requiring disallowed direction & & \\via slack & 4043 & 195,000 \\ \hline
%    \end{tabular}
%  \end{table}
  \begin{itemize}
    \item[] Link types that allowed both directions:
      \begin{itemize}
        \item[] 7 / 113 = 6.19\%
      \end{itemize}
    \item[] Link tokens that required disallowed direction via slack:
      \begin{itemize}
      \item[] 4043 / 195,000 = 2.07\%
      \end{itemize}
  \end{itemize}
  \begin{itemize}
    \uncover<1>{\item[]}
    \uncover<2->{\item[] The link labels mostly have a consistent direction.}
  \end{itemize}
}




\frame{\frametitle{Link Results: Subject-Verb links are backwards}
\begin{figure}
  \begin{dependency}
	\begin{deptext}
	  - \& n-u \& v \& e \& e \& v \& v-d \& r \& n-u \& - \\
	  the \& matter \& may \& never \& even \& be \& tried \& in \& court \& . \\
	  {\scriptsize DT} \& {\scriptsize NN} \& {\scriptsize MD} \& {\scriptsize RB} \& {\scriptsize RB} \& {\scriptsize VB} \& {\scriptsize VB} \& {\scriptsize IN} \& {\scriptsize NN} \& {\scriptsize .} \\
	\end{deptext}
	\deproot[edge below, edge style = {blue, dotted}]{3}{\small ROOT}
	\depedge[edge above, edge style = {red, ultra thick}]{2}{3}{S}
	\depedge[edge below, edge style = {blue, ultra thick}]{3}{2}{\small SBJ}
	\depedge[edge below, edge style = {blue, dotted}]{3}{4}{\small ADV}
	\depedge[edge below, edge style = {blue, dotted}]{3}{5}{\small ADV}
	\depedge[edge above, edge style = {red, thick}]{3}{6}{I}
	\depedge[edge below, edge style = {blue, thick}]{3}{6}{\small VC}
	\depedge[edge below, edge style = {blue, dotted}, edge unit distance =1.5ex]{3}{10}{\small P}
	\depedge[edge above, edge style = {red, thick}]{2}{1}{D}
	\depedge[edge below, edge style = {blue, thick}]{2}{1}{\small NMOD}
	\depedge[edge above, edge style = {red, thick}]{7}{8}{MV}
	\depedge[edge below, edge style = {blue, thick}]{7}{8}{\small ADV}
	\depedge[edge above, edge style = {orange, thick}]{6}{7}{P}
	\depedge[edge below, edge style = {blue, thick}]{6}{7}{\small VC}
	\depedge[edge above, edge style = {red, thick}]{8}{9}{J}
	\depedge[edge below, edge style = {blue, thick}]{8}{9}{\small PMOD}
	\deproot[edge above, edge style = {red, dotted}]{2}{W}
	\deproot[edge above, edge style = {orange, ultra thick, dotted}]{7}{WV}
	\deproot[edge above, edge style = {red, dotted}]{10}{X}
	\depedge[edge above, edge style = {red, dotted}]{6}{4}{E}
	\depedge[edge above, edge style = {red, dotted}]{6}{5}{E}
  \end{dependency}
\end{figure}

}



\frame{\frametitle{Link Results: Subject-Verb links are backwards}
\begin{figure}
  \begin{dependency}
	\begin{deptext}
	  - \& n-u \& v \& e \& e \& v \& v-d \& r \& n-u \& - \\
	  the \& matter \& may \& never \& even \& be \& tried \& in \& court \& . \\
	  {\scriptsize DT} \& {\scriptsize NN} \& {\scriptsize MD} \& {\scriptsize RB} \& {\scriptsize RB} \& {\scriptsize VB} \& {\scriptsize VB} \& {\scriptsize IN} \& {\scriptsize NN} \& {\scriptsize .} \\
	\end{deptext}
	\deproot[edge below, edge style = {gray}, hide label]{3}{\small ROOT}
	\depedge[edge above, edge style = {red, ultra thick}]{2}{3}{S}
	\depedge[edge below, edge style = {blue, ultra thick}]{3}{2}{\small SBJ}
	\depedge[edge below, edge style = {gray}, hide label]{3}{4}{\small ADV}
	\depedge[edge below, edge style = {gray}, hide label]{3}{5}{\small ADV}
	\depedge[edge above, edge style = {gray}, hide label]{3}{6}{I}
	\depedge[edge below, edge style = {gray}, hide label]{3}{6}{\small VC}
	\depedge[edge below, , edge style = {gray}, hide label, edge unit distance =1.5ex]{3}{10}{\small P}
	\depedge[edge above, edge style = {gray}, hide label]{2}{1}{D}
	\depedge[edge below, edge style = {gray}, hide label]{2}{1}{\small NMOD}
	\depedge[edge above, edge style = {gray}, hide label]{7}{8}{MV}
    \depedge[edge below, edge style = {gray}, hide label]{7}{8}{\small ADV}
	\depedge[edge above, edge style = {gray}, hide label]{6}{7}{P}
	\depedge[edge below, edge style = {gray}, hide label]{6}{7}{\small VC}
	\depedge[edge above, edge style = {gray}, hide label]{8}{9}{J}
	\depedge[edge below, edge style = {gray}, hide label]{8}{9}{\small PMOD}
	\deproot[edge above, edge style = {gray}, hide label]{2}{W}
    \deproot[edge above, edge style = {gray}, hide label]{7}{WV}
	\deproot[edge above, edge style = {gray}, hide label]{10}{X}
	\depedge[edge above, edge style = {gray}, hide label]{6}{4}{E}
	\depedge[edge above, edge style = {gray}, hide label]{6}{5}{E}
  \end{dependency}
\end{figure}

}


\frame{\frametitle{Link Results: Subject-Verb links are backwards}

\begin{itemize}
\item This is due to an inconsistency of the Link Grammar, discovered by our method. 
\end{itemize}

\begin{figure}
  \begin{dependency}
	\begin{deptext}
	  Jill \& wanted \& to \& skip \\
	\end{deptext}
    \deproot[edge above, edge unit distance=2ex]{1}{W}
    \deproot[edge above, edge unit distance=2ex]{2}{WV}
	\depedge[edge above]{2}{1}{S}
	\depedge[edge above]{2}{3}{TO}
	\depedge[edge above]{3}{4}{I}
	\depedge[edge above]{2}{4}{IV}
  \end{dependency}
\end{figure}

\begin{figure}
  \begin{dependency}
	\begin{deptext}
	  Jill \& persuaded \& him \& to \& skip \\
	\end{deptext}
    \deproot[edge above, edge unit distance=2ex]{1}{W}
    \deproot[edge above, edge unit distance=2ex]{2}{WV}
	\depedge[edge above]{2}{1}{S}
	\depedge[edge above]{2}{3}{O}
	\depedge[edge above]{2}{4}{TO}
	\depedge[edge above]{4}{5}{I}
	\depedge[edge above]{2}{5}{IV}
  \end{dependency}
\end{figure}




\NoteJH{Should I include a section of the appendix table? Which rows should I include?}
}


\section{Conclusions}
\frame{\frametitle{Conclusions}
  \begin{itemize}
  \item Link Grammar parses can be oriented into connected DAGs
  \item New corpora for building multi-headed dependency parsers
  \item ILP can be used to build or annotate corpora
  \end{itemize}
}


\frame{\frametitle{}
  \begin{itemize}
  \item[] Questions?
  \end{itemize}
}



%\nocite{*}
%\bibliographystyle{plain}
%\bibliography{hong+eisner.TLT13.slides.bib}
%\appendix

\end{document}
