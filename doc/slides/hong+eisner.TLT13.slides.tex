\documentclass{beamer}

% For more theme options:
%/usr/share/texmf/tex/latex/beamer/base/themes/theme/compatibility
\usepackage{beamerthemelined}

% this package seems to throw an error for me. -Juneki 12/6/14
%\usepackage[usenames,dvipsnames,svgnames,table]{xcolor}
\usepackage{soul}

\usepackage{algorithm}
\usepackage[noend]{algpseudocode}
\usepackage{graphicx}
\usepackage{caption}
\usepackage{subcaption}
\usepackage{longtable}
\usepackage{pdfpages}

\usepackage{tikz-dependency}

\newcommand{\eqnref}[1]{\eqref{eqn:#1}}
%\usepackage[usenames,dvipsnames,svgnames,table]{xcolor}  % allows better color names
\usepackage{todonotes}   % insert [disable] to disable all notes
\newcommand{\Note}[4][]{\todo[author=#2,color=#3,fancyline,#1]{#4}}
\newcommand{\noteJH}[2][]{\Note[#1]{JH}{blue!40}{#2}}
\newcommand{\noteJE}[2][]{\Note[#1]{JE}{green!40}{#2}}
\newcommand{\notewho}[3][]{\Note[#1]{#2}{orange!40}{#3}}  % extra arg with miscellaneous author
\newcommand{\NoteJH}[2][]{\noteJH[inline,#1]{#2}}
\newcommand{\NoteJE}[2][]{\noteJE[inline,#1]{#2}}
\newcommand{\Notewho}[3][]{\notewho[inline,#1]{#2}{#3}}  % extra arg with miscellaneous author


\setbeamerfont{page number in head/foot}{}
\setbeamertemplate{footline}[frame number]

\begin{document}
\title{Deriving Multi-Headed Planar Dependency Parses from Link Grammar Parses}
\author{Juneki Hong and Jason Eisner}
\date{}%\today}
\frame{\titlepage} %\NoteJH{Include pictures of us?}

\section{Introduction}
\frame{\frametitle{Introduction}
\begin{itemize}
\item This talk is about converting from one annotation style to another.
\item The conversion could be hard, where information is fragmented, missing, or ambiguous.

%This is often useful.  You may want to do it yourself.  
\item We use a general technique, Integer Linear Programming to help us do this conversion.
\end{itemize}
}

\frame{\frametitle{In Our Case: What We Started With}
  \begin{figure}
    \begin{dependency}[edge style={-}]
	  \begin{deptext}
	    - \& n-u \& v \& e \& e \& v \& v-d \& r \& n-u \& - \\
	    the \& matter \& may \& never \& even \& be \& tried \& in \& court \& . \\
	  \end{deptext}
	  \depedge[edge above, thick, edge style = {red}]{2}{3}{S}
	  \depedge[edge above, thick, edge style = {red}]{3}{6}{I}
	  \depedge[edge above, thick, edge style = {red}]{2}{1}{D}
	  \depedge[edge above, thick, edge style = {red}]{7}{8}{MV}
	  \depedge[edge above, thick, edge style = {red}]{6}{7}{P}
	  \depedge[edge above, thick, edge style = {red}]{8}{9}{J}
	  \deproot[edge above, thick, edge style = {red}]{2}{W}
	  \deproot[edge above, thick, edge style = {red}]{7}{WV}
	  \deproot[edge above, thick, edge style = {red}]{10}{X}
	  \depedge[edge above, thick, edge style = {red}]{6}{4}{E}
	  \depedge[edge above, thick, edge style = {red}]{6}{5}{E}
    \end{dependency}
    \caption*{Link Grammar: Parse with undirected edges}
  \end{figure}
}

\frame{\frametitle{What We Wanted:}
  \begin{figure}
    \begin{dependency}
	  \begin{deptext}
	    - \& n-u \& v \& e \& e \& v \& v-d \& r \& n-u \& - \\
	    the \& matter \& may \& never \& even \& be \& tried \& in \& court \& . \\
	  \end{deptext}
	  \depedge[edge above, thick, edge style = {red}]{2}{3}{S}
	  \depedge[edge above, thick, edge style = {red}]{3}{6}{I}
	  \depedge[edge above, thick, edge style = {red}]{2}{1}{D}
	  \depedge[edge above, thick, edge style = {red}]{7}{8}{MV}
	  \depedge[edge above, thick, edge style = {red}]{6}{7}{P}
	  \depedge[edge above, thick, edge style = {red}]{8}{9}{J}
	  \deproot[edge above, thick, edge style = {red}]{2}{W}
	  \deproot[edge above, thick, edge style = {red}]{7}{WV}
	  \deproot[edge above, thick, edge style = {red}]{10}{X}
	  \depedge[edge above, thick, edge style = {red}]{6}{4}{E}
	  \depedge[edge above, thick, edge style = {red}]{6}{5}{E}
    \end{dependency}
    \caption*{Multiheaded parse with directionalized edges}
  \end{figure} 

}

\frame{\frametitle{Why We Wanted That}
  \begin{itemize}
    \item We want to develop parsing algorithms for parses that look like this
    \item We couldn't figure out where to get the data to test them.
  \end{itemize}
}




\frame{\frametitle{Single-headedness}
  \begin{itemize}
  \item Dependency parse treebanks today are either single-headed or not planar.
  \item Stanford Dependencies are multiheaded but not planar
  \end{itemize}

\begin{figure}
  \begin{dependency}
	\begin{deptext}
	  {\scriptsize DT} \& {\scriptsize NN} \& {\scriptsize MD} \& {\scriptsize RB} \& {\scriptsize RB} \& {\scriptsize VB} \& {\scriptsize VB} \& {\scriptsize IN} \& {\scriptsize NN} \& {\scriptsize .} \\
	  the \& matter \& may \& never \& even \& be \& tried \& in \& court \& . \\
	\end{deptext}
	\deproot[edge above, thick, edge style = {blue}]{3}{\small ROOT}
	\depedge[edge above, thick, edge style = {blue}]{3}{2}{\small SBJ}
	\depedge[edge above, thick, edge style = {blue}]{3}{4}{\small ADV}
	\depedge[edge above, thick, edge style = {blue}]{3}{5}{\small ADV}
	\depedge[edge above, thick, edge style = {blue}]{3}{6}{\small VC}
	\depedge[edge above, thick, edge style = {blue}, edge unit distance =1.5ex]{3}{10}{\small P}
	\depedge[edge above, thick, edge style = {blue}]{2}{1}{\small NMOD}
	\depedge[edge above, thick, edge style = {blue}]{7}{8}{\small ADV}
	\depedge[edge above, thick, edge style = {blue}]{6}{7}{\small VC}
	\depedge[edge above, thick, edge style = {blue}]{8}{9}{\small PMOD}
  \end{dependency}
  \caption*{Some example dependency parse.}
\end{figure}

\uncover<2->{\begin{center}
    Link Grammar is almost a multiheaded planar corpora! We just need to directionalize the links.
\end{center}}
}



\section{Motivation, Overview}
%\subsection{Multi-headed}
\frame{\frametitle{Why Multi-headedness?}
  Multi-headedness Can Capture Additional Linguistic Phenomenon
  \begin{itemize}
    \item Control
    \item Relativization
    \item Conjunction
  \end{itemize}
}

\subsection{Control, Relativization, Conjunction}
\frame{\frametitle{Control}
  \begin{figure}
    \begin{dependency}
      \begin{deptext}
        Jill \& likes \& to \& skip \\
      \end{deptext}
      \deproot[edge above, thick, hide label, edge unit distance = 1.5ex]{2}{}
      \depedge[edge above, thick, hide label]{2}{1}{}
      \depedge[edge below, ultra thick, hide label, edge style = {purple}, edge unit distance = 1.2ex]{4}{1}{}
      \depedge[edge above, thick, hide label]{2}{3}{}
      \depedge[edge above, thick, hide label]{3}{4}{}
    \end{dependency}
    \caption*{Jill is the subject of two verbs}
  \end{figure}

  \begin{figure}
    \begin{dependency}
      \begin{deptext}
        Jill \& persuaded \& Jack \& to \& skip \\
      \end{deptext}
      \deproot[edge above, thick, hide label, edge unit distance = 1.5ex]{2}{}
      \depedge[edge above, thick, hide label]{2}{1}{}
      \depedge[edge above, thick, hide label]{2}{3}{}
      \depedge[edge above, thick, hide label]{3}{4}{}
      \depedge[edge above, thick, hide label]{4}{5}{}
      \depedge[edge below, ultra thick, hide label, edge style = {purple}, edge unit distance = 1.5ex]{5}{3}{}
      \depedge[edge below, ultra thick, hide label, edge style = {purple}, edge unit distance = 1.5ex]{2}{5}{}
    \end{dependency}
    \caption*{Jack is the object of one verb and the subject of another}
  \end{figure}
}


%\subsection{Relativization}
\frame{\frametitle{Relativization}
  \begin{figure}
    \begin{dependency}
      \begin{deptext}
        The \& boy \& that \& Jill \& skipped \& with \& fell \& down \\
      \end{deptext}
      \deproot[edge above, thick, hide label, edge unit distance = 2ex]{7}{}
      \depedge[edge above, thick, hide label]{2}{1}{}
      \depedge[edge above, thick, hide label]{2}{3}{}
      \depedge[edge above, thick, hide label]{3}{5}{}
      \depedge[edge above, thick, hide label]{5}{4}{}
      \depedge[edge above, thick, hide label]{5}{6}{}
      \depedge[edge above, thick, hide label, edge unit distance = 1.8ex]{7}{2}{}
      \depedge[edge above, thick, hide label]{7}{8}{}
      \depedge[edge below, ultra thick, hide label, edge style = {purple}, edge unit distance = 1.5ex]{6}{2}{}
    \end{dependency}
    \caption*{The boy is the object of \textit{with} as well as the subject of \textit{fell}.}
  \end{figure}
}


%\subsection{Conjunction}
\frame{\frametitle{Conjunction}
  \begin{figure}
    \begin{dependency}
      \begin{deptext}
        Jack \& and \& Jill \& went \& up \& the \& hill \\
      \end{deptext}
      \deproot[edge above, thick, hide label, edge unit distance = 2ex]{4}{}
      \depedge[edge above, thick, hide label]{2}{1}{}
      \depedge[edge above, thick, hide label]{2}{3}{}
      \depedge[edge above, thick, hide label]{4}{2}{}
      \depedge[edge above, thick, hide label]{4}{5}{}
      \depedge[edge above, thick, hide label]{5}{7}{}
      \depedge[edge above, thick, hide label]{7}{6}{}

      \depedge[edge below, ultra thick, hide label, edge style = {purple}, edge unit distance = 1.5ex]{4}{1}{}
      \depedge[edge below, ultra thick, hide label, edge style = {purple}, edge unit distance = 1.5ex]{4}{3}{}

    \end{dependency}
    \caption*{Jack and Jill serve as the two arguments to \textit{and}, but are also subjects of \textit{went}.}
  \end{figure}
}



\subsection{Motivation}
\frame{\frametitle{Motivation}
  \begin{itemize}
    \uncover<1->{\item A multiheaded dependency corpus would be useful for testing new parsing algorithms}
    \uncover<2->{\item Such a corpus could be automatically annotated using Integer Linear Programming}
    \uncover<3->{\item We explored whether the Link Grammar could be adapted for this purpose.}
    \uncover<4->{\item The results of this are mixed, but provides a good case study.}
  \end{itemize}
}

\subsection{Corpus Building}
\frame{\frametitle{Corpus Building Strategy}
\begin{itemize}
\item We start with some sentences and parse them with LG Parser
\item We take the undirected parses and try to directionalize them.
\item We use an ILP to assign consistent directions for each link type.
\end{itemize}

\begin{figure}
  \includegraphics[width=\linewidth]{images/corpusbuilding}
\end{figure}
}



% Shouldn't mention non-projectivity. 
%\frame{\frametitle{Non-projectivity}
%\begin{figure}
%  \includegraphics[width=\linewidth]{multiheaded}
%  \caption*{English is not a completely projective language}
%  \NoteJH{image taken from Chris Dyer's slide: http://demo.clab.cs.cmu.edu/fa2014-11711/images/0/0e/Depparsing.pdf}
%  \NoteJH{I should also ask Chris for permission to use this}
%\end{figure}
%}


%\section{Non Projectivity}
%\frame{\frametitle{We do not address: Non-Projectivity}
%\begin{itemize}
%  \item Projectivity assumptions allow us to have efficient parsing algorithms that utilize dynamic programming
%  \item English is mostly projective
%\end{itemize}

%\NoteJH{Should I mention non-projectivity? This slide is mostly here because of a reviewer comment.}
%}

%\frame{\frametitle{We do not address: Non-Projectivity}
%\begin{itemize}
%  \item Projectivity assumptions allow us to have efficient parsing algorithms that utilize dynamic programming
%  \item English is mostly projective
%\end{itemize}

%\begin{figure}
%  \includegraphics[width=0.7\linewidth]{multiheaded}
%  \caption*{But not always}
%  \NoteJH{image taken from Chris Dyer's slide: http://demo.clab.cs.cmu.edu/fa2014-11711/images/0/0e/Depparsing.pdf}
%  \NoteJH{I should also ask Chris for permission to use this}
%\end{figure}


%}


\section{Link Grammars}

\frame{\frametitle{Link Grammars}
  \begin{itemize}
  \item[] Grammar-based formalism for projective dependency parsing with undirected links
  \item[] Original formalism and English Link Grammar created by Davy Temperley, Daniel Sleator, and John Lafferty (1991)
  \end{itemize}
}

\subsection{Link Grammars Intro}
\frame{\frametitle{Link Grammars: How They Work}
  \includegraphics[width=\linewidth]{linkgrammar_example1.png}\footnote{These figures were clipped from the original Link Grammar paper: \\``Parsing English with a Link Grammar'' by Sleator and Temperley}
}

\frame{\frametitle{Link Grammars: How They Work}
  \includegraphics[width=\linewidth]{linkgrammar_example2.png}
}

\frame{\frametitle{Link Grammars: How They Work}
  \includegraphics[width=\linewidth]{linkgrammar_example3.png}
}



\frame{\frametitle{Link Grammars: Same Example Parse From Before Again}
  \begin{figure}
    \begin{dependency}[edge style={-}]
	  \begin{deptext}
	    - \& n-u \& v \& e \& e \& v \& v-d \& r \& n-u \& - \\
	    the \& matter \& may \& never \& even \& be \& tried \& in \& court \& . \\
	  \end{deptext}
	  \depedge[edge above, thick, edge style = {red}]{2}{3}{S}
	  \depedge[edge above, thick, edge style = {red}]{3}{6}{I}
	  \depedge[edge above, thick, edge style = {red}]{2}{1}{D}
	  \depedge[edge above, thick, edge style = {red}]{7}{8}{MV}
	  \depedge[edge above, thick, edge style = {red}]{6}{7}{P}
	  \depedge[edge above, thick, edge style = {red}]{8}{9}{J}
	  \deproot[edge above, thick, edge style = {red}]{2}{W}
	  \deproot[edge above, thick, edge style = {red}]{7}{WV}
	  \deproot[edge above, thick, edge style = {red}]{10}{X}
	  \depedge[edge above, thick, edge style = {red}]{6}{4}{E}
	  \depedge[edge above, thick, edge style = {red}]{6}{5}{E}
    \end{dependency}
    \caption*{Link Parse of a sentence from Penn Tree Bank}
  \end{figure}
}

%\frame{\frametitle{Link Grammars: Converting into a Directed Acyclic Graph}
%  \begin{figure}
%    \begin{dependency}
%	  \begin{deptext}
%	    - \& n-u \& v \& e \& e \& v \& v-d \& r \& n-u \& - \\
%	    the \& matter \& may \& never \& even \& be \& tried \& in \& court \& . \\
%	  \end{deptext}
%	  \depedge[edge above, thick, edge style = {red}]{2}{3}{S}
%	  \depedge[edge above, thick, edge style = {red}]{3}{6}{I}
%	  \depedge[edge above, thick, edge style = {red}]{2}{1}{D}
%	  \depedge[edge above, thick, edge style = {red}]{7}{8}{MV}
%	  \depedge[edge above, thick, edge style = {red}]{6}{7}{P}
%	  \depedge[edge above, thick, edge style = {red}]{8}{9}{J}
%	  \deproot[edge above, thick, edge style = {red}]{2}{W}
%	  \deproot[edge above, thick, edge style = {red}]{7}{WV}
%	  \deproot[edge above, thick, edge style = {red}]{10}{X}
%	  \depedge[edge above, thick, edge style = {red}]{6}{4}{E}
%	  \depedge[edge above, thick, edge style = {red}]{6}{5}{E}
%    \end{dependency}
%    
%    \caption*{Directionalize the edges}
%  \end{figure} 
%}


\frame{\frametitle{Link Grammars}
Compare resulting dependency parse with CoNLL 2007 shared task. 
\begin{figure}
  \begin{dependency}
	\begin{deptext}
	  - \& n-u \& v \& e \& e \& v \& v-d \& r \& n-u \& - \\
	  the \& matter \& may \& never \& even \& be \& tried \& in \& court \& . \\
	  {\scriptsize DT} \& {\scriptsize NN} \& {\scriptsize MD} \& {\scriptsize RB} \& {\scriptsize RB} \& {\scriptsize VB} \& {\scriptsize VB} \& {\scriptsize IN} \& {\scriptsize NN} \& {\scriptsize .} \\
	\end{deptext}
	\deproot[edge below, thick, edge style = {blue}]{3}{\small ROOT}
	\depedge[edge above, thick, edge style = {red}]{2}{3}{S}
	\depedge[edge below, thick, edge style = {blue}]{3}{2}{\small SBJ}
	\depedge[edge below, thick, edge style = {blue}]{3}{4}{\small ADV}
	\depedge[edge below, thick, edge style = {blue}]{3}{5}{\small ADV}
	\depedge[edge above, thick, edge style = {red}]{3}{6}{I}
	\depedge[edge below, thick, edge style = {blue}]{3}{6}{\small VC}
	\depedge[edge below, thick, edge style = {blue}, edge unit distance =1.5ex]{3}{10}{\small P}
	\depedge[edge above, thick, edge style = {red}]{2}{1}{D}
	\depedge[edge below, thick, edge style = {blue}]{2}{1}{\small NMOD}
	\depedge[edge above, thick, edge style = {red}]{7}{8}{MV}
	\depedge[edge below, thick, edge style = {blue}]{7}{8}{\small ADV}
	\depedge[edge above, thick, edge style = {red}]{6}{7}{P}
	\depedge[edge below, thick, edge style = {blue}]{6}{7}{\small VC}
	\depedge[edge above, thick, edge style = {red}]{8}{9}{J}
	\depedge[edge below, thick, edge style = {blue}]{8}{9}{\small PMOD}
	\deproot[edge above, thick, edge style = {red}]{2}{W}
	\deproot[edge above, thick, edge style = {red}]{7}{WV}
	\deproot[edge above, thick, edge style = {red}]{10}{X}
	\depedge[edge above, thick, edge style = {red}]{6}{4}{E}
	\depedge[edge above, thick, edge style = {red}]{6}{5}{E}
  \end{dependency}

  \caption*{Bottom half is CoNLL. Top half is the directionalized link parse.}
\end{figure}


}



\frame{\frametitle{Link Grammars}
Compare resulting dependency parse with CoNLL 2007 shared task. 
\begin{figure}
  \begin{dependency}
	\begin{deptext}
	  - \& n-u \& v \& e \& e \& v \& v-d \& r \& n-u \& - \\
	  the \& matter \& may \& never \& even \& be \& tried \& in \& court \& . \\
	  {\scriptsize DT} \& {\scriptsize NN} \& {\scriptsize MD} \& {\scriptsize RB} \& {\scriptsize RB} \& {\scriptsize VB} \& {\scriptsize VB} \& {\scriptsize IN} \& {\scriptsize NN} \& {\scriptsize .} \\
	\end{deptext}
	\deproot[edge below, edge style = {blue, dotted}]{3}{\small ROOT}
	\depedge[edge above, edge style = {red, ultra thick}]{2}{3}{S}
	\depedge[edge below, edge style = {blue, ultra thick}]{3}{2}{\small SBJ}
	\depedge[edge below, edge style = {blue, dotted}]{3}{4}{\small ADV}
	\depedge[edge below, edge style = {blue, dotted}]{3}{5}{\small ADV}
	\depedge[edge above, edge style = {red, thick}]{3}{6}{I}
	\depedge[edge below, edge style = {blue, thick}]{3}{6}{\small VC}
	\depedge[edge below, edge style = {blue, dotted}, edge unit distance =1.5ex]{3}{10}{\small P}
	\depedge[edge above, edge style = {red, thick}]{2}{1}{D}
	\depedge[edge below, edge style = {blue, thick}]{2}{1}{\small NMOD}
	\depedge[edge above, edge style = {red, thick}]{7}{8}{MV}
	\depedge[edge below, edge style = {blue, thick}]{7}{8}{\small ADV}
	\depedge[edge above, edge style = {orange, thick}]{6}{7}{P}
	\depedge[edge below, edge style = {blue, thick}]{6}{7}{\small VC}
	\depedge[edge above, edge style = {red, thick}]{8}{9}{J}
	\depedge[edge below, edge style = {blue, thick}]{8}{9}{\small PMOD}
	\deproot[edge above, edge style = {red, dotted}]{2}{W}
	\deproot[edge above, edge style = {orange, ultra thick, dotted}]{7}{WV}
	\deproot[edge above, edge style = {red, dotted}]{10}{X}
	\depedge[edge above, edge style = {red, dotted}]{6}{4}{E}
	\depedge[edge above, edge style = {red, dotted}]{6}{5}{E}
  \end{dependency}

  \caption*{Bottom half is CoNLL. Top half is the directionalized link parse.}
\end{figure}


}


%\section{ILP}
%\frame{\frametitle{Integer Linear Programming}

%\NoteJH{Will the audience know about ILP?}

%}


\section{ILP Model}

\subsection{What is ILP?}
\frame{\frametitle{What is Integer Linear Programming?}
\begin{itemize}
  \item An optimization problem where some or all of the variables are integers. 
  \item The objective function and constraints are linear.
  \item In general, it's NP-Hard! But good solvers exist that work well most of the time.
  \item Our ILP is encoded as a ZIMPL program and solved using the SCIP Optimization Suite\footnote{\url{http://scip.zib.de/}} 
\end{itemize}
}

\subsection{ILP Model}
\frame{\frametitle{Integer Linear Programming Model}
Encoded Constraints:
\begin{itemize}
  \item Acyclicity\uncover<2->{: (\textbf{No cycles!})}
  \item Connectedness\uncover<3->{: (\textbf{Every word is reachable from a root})}
  \item Consistency of Directionalized Links\uncover<4->{:

    (\textbf{Similar links oriented the same way})}
\end{itemize}

\begin{figure}
  \begin{subfigure}[b]{0.3\textwidth}
    \uncover<2->{\includegraphics[width=\linewidth]{images/acyclic2}}
  \end{subfigure}
  \begin{subfigure}[b]{0.3\textwidth}
    \uncover<3->{\includegraphics[width=\linewidth]{images/connectedness}}
  \end{subfigure}
  \begin{subfigure}[b]{0.3\textwidth}
    \uncover<4->{\includegraphics[width=\linewidth]{images/consistency}}
  \end{subfigure}
  
\end{figure}


}




\frame{\frametitle{Integer Linear Programming Model}

For each sentence, for each edge $i,j$, where $i < j$

\begin{figure}
  \begin{dependency}[edge style={-}]
    \begin{deptext}
      . \& . \& . \& $i$ \& . \& . \& . \& $j$ \& . \& . \& . \\
    \end{deptext}
    \depedge[edge above, thick, edge unit distance = 1.1ex]{4}{8}{L}
  \end{dependency}
\end{figure}



Variables: 
\begin{itemize}
\item[] $x_{ij}, x_{ji} \in \mathbb{Z} \geq 0$: orientation of each link
\item[] $x_{ij} + x_{ji} = 1$
\end{itemize}
\uncover<2->{\begin{center}
    \textbf{An individual link token can either be oriented left or oriented right}
\end{center}}

}



\frame{\frametitle{Acyclicity, Connectedness}
\begin{itemize}
  \item[] Acyclicity
    \begin{itemize}
      \item[] Given that node $u$ is the parent of $v$
      \item[] $n_v$: length of the sentence containing node $v$
      \item[] $d_v \in [0, n_v]$: depth of the node from the root of the sentence
    \end{itemize}

    \begin{align}
      (\forall_u)\; d_v + (1 + n_v) \cdot (1 - x_{uv}) & \geq 1+d_u
    \end{align}
    \uncover<2->{\begin{center}
        \textbf{The depth of a child is greater than the depth of the parent}
    \end{center}}

  \item[] Connectedness
    \begin{align}
      \sum_u x_{uv} & \geq 1 
    \end{align}
    \uncover<3->{\begin{center}
        \textbf{A word has at least 1 parent}
    \end{center}}
\end{itemize}
}


\frame{\frametitle{Consistency of Directionalized Links \onslide<2->{with Slack}}
\begin{itemize}
  \item[] Consistency of Directionalized Links
    \begin{itemize}
      \item[] $r_L, \ell_L \in \{0,1\}$: whether all links with label $L$ allowed left/right
    \end{itemize}
    
    \begin{align}\label{direction+slack}
      x_{ij} &\leq r_L \onslide<2->{ + s_{ij}} &
      x_{ji} &\leq \ell_L \onslide<2->{ + s_{ij}}
    \end{align}

    Objective Function:
    \begin{align}\label{eqn:obj}
      \min \left( \sum_L r_L + \ell_L \right) \onslide<2->{\cdot \frac{N_L}{4} + \sum_{ij}s_{ij}}
    \end{align}

    \onslide<2->{\begin{itemize}
      \item[] $s_{ij} \in \mathbb{R} \geq 0$: slack variable 
      \item[] $N_L$: Number of link tokens with label $L$
    \end{itemize}
    }
    \onslide<3->{
    \begin{itemize}
      \item[] \textbf{Slack allows a few links with label $L$ in disallowed directions}
    \end{itemize}
    }

\end{itemize}
}


\section{Experiments and Results}
%\frame{\frametitle{Experiments and Results}
%\begin{itemize}
%  \item 18,577 English sentences with gold CoNLL. 
%  \item 18,577 unlabeled Russian sentences.
%\end{itemize}
%}

\frame{\frametitle{Data Sets}
%\begin{itemize}
%  \item[] 18,577 English sentences with 10,960 connected parses
%  \item[] 18,577 unlabeled Russian sentences with 4,913 connected parses
%\end{itemize}
  
  Data Sets taken from:
  \begin{itemize}
    \item[] CoNLL 2007 Shared Task (English)
    \item[] ACL 2013 Shared Task of Machine Translation (Russian)
  \end{itemize}

  \begin{figure}[h]
    \centering
    \begin{tabular}{|l|l|l|}
      \hline
       & Input Sentences & Output Connected Parses \\ \hline
       English  & 18,577        & 10,960               \\ \hline
       Russian & 18,577        & 4,913                \\ \hline
    \end{tabular}
  \end{figure}
}

\frame{\frametitle{Stability of Results}

  \begin{itemize}
  \item We were worried that the recovered direction mapping might be unstable and sensitive to the input corpus. 
  \item We compared the results of increasing runs of sentences.
  \end{itemize}

  \begin{figure}
    \centering
    \includegraphics[width=0.6\linewidth]{precision_recall}
  \end{figure}
}



%\frame{\frametitle{On the English data set}
%\begin{itemize}
%\item[] Link Data has 8\% additional edges over the CoNLL.
%\item[] 52\% of links match CoNLL arcs
%\item[] 57\% of CoNLL arcs match links
%\end{itemize}

%
%}

\frame{\frametitle{On the English Data Set:}
  
  \uncover<2->{Multiheadedness}
  \begin{itemize}
  \uncover<2->{\item[] Link Data has 8\% additional edges over the CoNLL. 

    (average about 2 multiheaded words per sentence)}
  \end{itemize}
  
  \uncover<3->{CoNLL Matches}
  \begin{itemize}
  \uncover<3->{\item[] 52\% of links match CoNLL arcs
  \item[] 57\% of CoNLL arcs match links}
  \end{itemize}

  \uncover<4->{Directionality}%\uncover<5->{ is Mostly Consistent!}
  \begin{itemize}
  \uncover<4->{\item[] 6.19\% of link types allowed both directions %($\frac{7}{113}$)
  \item[] 2.07\% of link tokens required disallowed direction via slack %($\frac{4043}{195,000}$)
  }
  \end{itemize}


  %\uncover<4->{The link labels mostly have a consistent direction.}
  %\uncover<5->{\item[] The direction is often the same as CoNLL when there is a corresponding arc}

}



\frame{\frametitle{ILP Results: Top 25 Most Occurring Labels}

  \begin{figure}
    \centering
    \input{link_table_filtered.tex}
  \end{figure}
}

\frame{\frametitle{ILP Results: Top 25 Most Occurring Labels}
  \begin{figure}
    \centering
    \begin{tiny}
\centering
%\begin{longtable}[width=\textwidth]{|l|l|l|l|l|l|}
\begin{tabular}{|l|l|l|l|l|l|}
\hline
Label & Rightward & Multiheaded & CoNLL Match & CoNLL Dir Match\\\hline
%\endhead

\hline
%\endfoot

A & 0\% (0/8501) & 0\% (0/8501) & 84\% (7148/8501) & 98\% (7002/7148)  \\
AN & 0\% (0/9401) & 0\% (0/9401) & 83\% (7825/9401) & 98\% (7639/7825)  \\
\textbf{B} & \textbf{100\% (1514/1515)} & \textbf{61\% (919/1515)} & \textbf{53\% (806/1515)} & \textbf{84\% (678/806)} \\
C & 100\% (3272/3272) & 0\% (0/3272) & 3\% (85/3272) & 53\% (45/85)  \\
CO & 0\% (0/2478) & 1\% (32/2478) & 5\% (114/2478) & 68\% (78/114)  \\
\textbf{CV} & \textbf{100\% (3237/3237)} & \textbf{100\% (3237/3237)} & \textbf{56\% (1827/3237)} & \textbf{28\% (512/1827)}  \\
D & 0\% (56/19535) & 0\% (71/19535) & 85\% (16656/19535) & 100\% (16608/16656)  \\
E & 0\% (0/1897) & 0\% (2/1897) & 67\% (1279/1897) & 99\% (1263/1279)  \\
G & 0\% (0/6061) & 0\% (0/6061) & 70\% (4258/6061) & 96\% (4070/4258)  \\
I & 100\% (5405/5424) & 60\% (3247/5424) & 95\% (5168/5424) & 47\% (2408/5168)  \\
IV & 100\% (1626/1627) & 100\% (1626/1627) & 85\% (1389/1627) & 97\% (1353/1389)  \\
J & 98\% (16400/16673) & 2\% (280/16673) & 87\% (14522/16673) & 97\% (14069/14522)  \\
M & 100\% (9594/9596) & 0\% (16/9596) & 74\% (7124/9596) & 92\% (6583/7124)  \\
MV & 100\% (13375/13376) & 0\% (61/13376) & 51\% (6797/13376) & 98\% (6681/6797)  \\
MX & 100\% (1999/1999) & 4\% (83/1999) & 42\% (836/1999) & 91\% (763/836)  \\
O & 100\% (11027/11028) & 0\% (0/11028) & 81\% (8932/11028) & 96\% (8535/8932)  \\
P & 100\% (3755/3756) & 31\% (1167/3756) & 94\% (3528/3756) & 100\% (3523/3528)  \\
S & 97\% (13138/13520) & 57\% (7662/13520) & 92\% (12476/13520) & 5\% (586/12476)  \\
SJ & 50\% (2736/5468) & 0\% (0/5468) & 69\% (3778/5468) & 93\% (3502/3778)  \\
TO & 100\% (1733/1734) & 0\% (1/1734) & 0\% (5/1734) & 100\% (5/5)  \\
VJ & 51\% (765/1500) & 1\% (8/1500) & 71\% (1059/1500) & 89\% (939/1059)  \\
W & 100\% (10528/10528) & 0\% (5/10528) & 5\% (504/10528) & 46\% (232/504)  \\
WV & 100\% (7563/7563) & 100\% (7557/7563) & 57\% (4345/7563) & 97\% (4214/4345)  \\
X & 80\% (13132/16406) & 5\% (806/16406) & 8\% (1364/16406) & 95\% (1300/1364)  \\
YS & 0\% (0/1645) & 0\% (0/1645) & 98\% (1619/1645) & 0\% (0/1619)  \\
\hline
%\end{longtable}
\end{tabular}
\end{tiny}

  \end{figure}
}

\frame{\frametitle{ILP Results: Top 25 Most Occurring Labels}
  \begin{figure}
    \centering
    \begin{tiny}
\centering
%\begin{longtable}[width=\textwidth]{|l|l|l|l|l|l|}
\begin{tabular}{|l|l|l|l|l|l|}
  \hline
  Label & Rightward & Multiheaded & CoNLL Match & CoNLL Dir Match\\\hline
  %\endhead
  
  \textbf{B} & \textbf{100\% (1514/1515)} & \textbf{61\% (919/1515)} & \textbf{53\% (806/1515)} & \textbf{84\% (678/806)} \\
  \hline
\end{tabular}
\end{tiny}

%\begin{small}
\begin{itemize}
  \uncover<2->{
  \item[] ``B'' link relative clauses
  }
\end{itemize}
%\end{small}

\begin{small}
\uncover<2->{
  \begin{subfigure}[b]{0.48\textwidth}
    \centering
  \begin{dependency}[edge style={-}]
    \begin{deptext}
      The \& dog \& I \& had \& chased \& was \& green \\
    \end{deptext}
    \depedge[edge above, edge style = {red, thick}]{2}{3}{R}
    \depedge[edge above, edge style = {red, thick}]{3}{4}{S}
    \depedge[edge above, edge style = {red, thick}]{4}{5}{PP}
    \depedge[edge above, edge style = {orange, ultra thick}]{2}{5}{B}
  \end{dependency}
\end{subfigure}
}
\uncover<3->{
  \begin{subfigure}[b]{0.48\textwidth}
    \centering
  \begin{dependency}[edge style={-}]
    \begin{deptext}
      I \& told \& him \& I \& had \& oranges \\
    \end{deptext}
    \depedge[edge above, edge style = {red, thick}, edge unit distance = 1.5ex]{2}{4}{CE}
    \depedge[edge above, edge style = {red, thick}]{4}{5}{S}
    \depedge[edge above, edge style = {orange, ultra thick}]{2}{5}{CV}
  \end{dependency}
\end{subfigure}
}

\end{small}


\begin{tiny}
\begin{tabular}{|l|l|l|l|l|l|}
  \hline
  Label & Rightward & Multiheaded & CoNLL Match & CoNLL Dir Match\\\hline
  \textbf{CV} & \textbf{100\% (3237/3237)} & \textbf{100\% (3237/3237)} & \textbf{56\% (1827/3237)} & \textbf{28\% (512/1827)}  \\
  \hline
  %\end{longtable}
\end{tabular}
\end{tiny}

%\begin{small}
\begin{itemize}
  \uncover<3->{
  \item[] ``CV'' link conjunctions to main verbs of clauses.
  }
\end{itemize}

%\end{small}

  \end{figure}
}


%\frame{\frametitle{ILP Results: Top 25 Most Occuring Labels}
%  \begin{tiny}
\centering
%\begin{longtable}[width=\textwidth]{|l|l|l|l|l|l|}
\begin{tabular}{|l|l|}
\hline
Label & CoNLL Label\\\hline
%\endhead

\hline
%\endfoot

A & NMOD 98\% (7000/7148) \\
AN & NMOD 96\% (7523/7825) \\
B & NMOD 75\% (603/806) \\
C & NMOD 27\% (23/85) \\
CO & NMOD 39\% (44/114) \\
CV & VMOD 52\% (956/1827) \\
D & NMOD 100\% (16629/16656) \\
E & ADV 84\% (1079/1279) \\
G & NMOD 91\% (3868/4258) \\
I & VMOD 53\% (2756/5168) \\
IV & OBJ 70\% (972/1389) \\
J & PMOD 98\% (14240/14522) \\
M & NMOD 82\% (5873/7124) \\
MV & ADV 75\% (5105/6797) \\
MX & NMOD 89\% (741/836) \\
O & OBJ 73\% (6495/8932) \\
P & VC 60\% (2104/3528) \\
S & SBJ 97\% (12150/12476) \\
SJ & COORD 86\% (3240/3778) \\
TO & ADV 40\% (2/5) \\
VJ & COORD 86\% (916/1059) \\
W & P 52\% (262/504) \\
WV & ROOT 97\% (4214/4345) \\
X & P 100\% (1358/1364) \\
YS & NMOD 100\% (1619/1619) \\
\hline
%\end{longtable}
\end{tabular}
\end{tiny}

%}



\subsection{Subject-Verb Links}
\frame{\frametitle{Link Results: Subject-Verb Links are Backwards}
\begin{figure}
  \begin{dependency}
	\begin{deptext}
	  - \& n-u \& v \& e \& e \& v \& v-d \& r \& n-u \& - \\
	  the \& matter \& may \& never \& even \& be \& tried \& in \& court \& . \\
	  {\scriptsize DT} \& {\scriptsize NN} \& {\scriptsize MD} \& {\scriptsize RB} \& {\scriptsize RB} \& {\scriptsize VB} \& {\scriptsize VB} \& {\scriptsize IN} \& {\scriptsize NN} \& {\scriptsize .} \\
	\end{deptext}
	\deproot[edge below, edge style = {blue, dotted}]{3}{\small ROOT}
	\depedge[edge above, edge style = {red, ultra thick}]{2}{3}{S}
	\depedge[edge below, edge style = {blue, ultra thick}]{3}{2}{\small SBJ}
	\depedge[edge below, edge style = {blue, dotted}]{3}{4}{\small ADV}
	\depedge[edge below, edge style = {blue, dotted}]{3}{5}{\small ADV}
	\depedge[edge above, edge style = {red, thick}]{3}{6}{I}
	\depedge[edge below, edge style = {blue, thick}]{3}{6}{\small VC}
	\depedge[edge below, edge style = {blue, dotted}, edge unit distance =1.5ex]{3}{10}{\small P}
	\depedge[edge above, edge style = {red, thick}]{2}{1}{D}
	\depedge[edge below, edge style = {blue, thick}]{2}{1}{\small NMOD}
	\depedge[edge above, edge style = {red, thick}]{7}{8}{MV}
	\depedge[edge below, edge style = {blue, thick}]{7}{8}{\small ADV}
	\depedge[edge above, edge style = {orange, thick}]{6}{7}{P}
	\depedge[edge below, edge style = {blue, thick}]{6}{7}{\small VC}
	\depedge[edge above, edge style = {red, thick}]{8}{9}{J}
	\depedge[edge below, edge style = {blue, thick}]{8}{9}{\small PMOD}
	\deproot[edge above, edge style = {red, dotted}]{2}{W}
	\deproot[edge above, edge style = {orange, ultra thick, dotted}]{7}{WV}
	\deproot[edge above, edge style = {red, dotted}]{10}{X}
	\depedge[edge above, edge style = {red, dotted}]{6}{4}{E}
	\depedge[edge above, edge style = {red, dotted}]{6}{5}{E}
  \end{dependency}
\end{figure}

}



\frame{\frametitle{Link Results: Subject-Verb Links are Backwards}
\begin{figure}
  \begin{dependency}
	\begin{deptext}
	  - \& n-u \& v \& e \& e \& v \& v-d \& r \& n-u \& - \\
	  the \& matter \& may \& never \& even \& be \& tried \& in \& court \& . \\
	  {\scriptsize DT} \& {\scriptsize NN} \& {\scriptsize MD} \& {\scriptsize RB} \& {\scriptsize RB} \& {\scriptsize VB} \& {\scriptsize VB} \& {\scriptsize IN} \& {\scriptsize NN} \& {\scriptsize .} \\
	\end{deptext}
	\deproot[edge below, edge style = {gray}, hide label]{3}{\small ROOT}
	\depedge[edge above, edge style = {red, ultra thick}]{2}{3}{S}
	\depedge[edge below, edge style = {blue, ultra thick}]{3}{2}{\small SBJ}
	\depedge[edge below, edge style = {gray}, hide label]{3}{4}{\small ADV}
	\depedge[edge below, edge style = {gray}, hide label]{3}{5}{\small ADV}
	\depedge[edge above, edge style = {gray}, hide label]{3}{6}{I}
	\depedge[edge below, edge style = {gray}, hide label]{3}{6}{\small VC}
	\depedge[edge below, , edge style = {gray}, hide label, edge unit distance =1.5ex]{3}{10}{\small P}
	\depedge[edge above, edge style = {gray}, hide label]{2}{1}{D}
	\depedge[edge below, edge style = {gray}, hide label]{2}{1}{\small NMOD}
	\depedge[edge above, edge style = {gray}, hide label]{7}{8}{MV}
    \depedge[edge below, edge style = {gray}, hide label]{7}{8}{\small ADV}
	\depedge[edge above, edge style = {gray}, hide label]{6}{7}{P}
	\depedge[edge below, edge style = {gray}, hide label]{6}{7}{\small VC}
	\depedge[edge above, edge style = {gray}, hide label]{8}{9}{J}
	\depedge[edge below, edge style = {gray}, hide label]{8}{9}{\small PMOD}
	\deproot[edge above, edge style = {gray}, hide label]{2}{W}
    \deproot[edge above, edge style = {gray}, hide label]{7}{WV}
	\deproot[edge above, edge style = {gray}, hide label]{10}{X}
	\depedge[edge above, edge style = {gray}, hide label]{6}{4}{E}
	\depedge[edge above, edge style = {gray}, hide label]{6}{5}{E}
  \end{dependency}
\end{figure}

}


\frame{\frametitle{Link Results: Subject-Verb Links are Backwards}

\begin{itemize}
\item This is due to a possible inconsistency of the Link Grammar, discovered by our method. 
\end{itemize}

%    +-------WV-------+--------CV-------+
%    +---Wd---+---Ss--+--Ce-+-Ss-+---I--+
%    |        |       |     |    |      |
%LEFT-WALL Jill.f thinks.v he will.v skip.v 

\begin{figure}
\vspace{-6pt}
\centering
  \begin{dependency}[edge style={-}]
	\begin{deptext}
	  Jill \& thinks \& he \& will \& skip\\
	\end{deptext}
    %\deproot[edge above, edge unit distance=2ex]{1}{W}
    %\deproot[edge above, edge unit distance=2ex]{2}{WV}
	\depedge[edge above]{1}{2}{S}
	\depedge[edge above]{2}{3}{C}
	\depedge[edge above]{3}{4}{S}
	\depedge[edge above]{4}{5}{I}
	\depedge[edge above]{2}{5}{CV}
  \end{dependency}
  %\caption*{Links with label I is forced to go right}
\end{figure}



%    +--------------WV-------------+
%    +---Wd---+---Ss--+--MVi-+--I--+
%    |        |       |      |     |
%LEFT-WALL Jill.f thinks.v to.r skip.v 


\begin{figure}
\vspace{-6pt}
\centering
  \begin{dependency}[edge style={-}]
	\begin{deptext}
	  Jill \& hopes \& to \& skip\\ % Originally: Jill thinks to skip. The sentence was altered to be clearer.
	\end{deptext}
    %\deproot[edge above, edge unit distance=2ex]{1}{W}
    %\deproot[edge above, edge unit distance=2ex]{2}{WV}
	\depedge[edge above]{1}{2}{S}
	\depedge[edge above]{2}{3}{MV}
	\depedge[edge above]{3}{4}{I}
  \end{dependency}
  %\caption*{Links with label I is forced to go right}
\end{figure}

}



%\frame{\frametitle{Possible simpler explanation}

%linkparser> 
%linkparser> Jill made him go
%Found 5 linkages (3 had no P.P. violations)
%Linkage 1, cost vector = (UNUSED=0 DIS= 0.00 LEN=6)
%
%    +-------WV-------+----I*j---+
%    +---Wd---+---Ss--+--Ox-+    |
%    |        |       |     |    |
%LEFT-WALL Jill.f made.v-d him go.v 

%\begin{figure}
%\vspace{-6pt}
%\centering
%  \begin{dependency}
%	\begin{deptext}
%	  Jill \& made \& him \& go \\
%	\end{deptext}
%    \deproot[edge above, edge unit distance=2ex]{1}{W}
%    \deproot[edge above, edge unit distance=2ex]{2}{WV}
%	\depedge[edge above]{1}{2}{S}
%	\depedge[edge above]{2}{4}{I}
%	\depedge[edge above]{2}{3}{O}
%  \end{dependency}
%  \caption*{Links with label I is forced to go right}
%\end{figure}


%\begin{figure}
%  \vspace{-12pt}
%  \begin{dependency}
%	\begin{deptext}
%	  Jill \& would \& go \\
%	\end{deptext}
%    \deproot[edge above, edge unit distance=2ex]{1}{W}
%    \deproot[edge above, edge unit distance=2ex]{3}{WV}
%	\depedge[edge above]{1}{2}{S}
%	\depedge[edge above]{2}{3}{I}
%  \end{dependency}
%  \caption*{If I is forced to go right, then S is as well}
%\end{figure}
%}


\frame{\frametitle{Link Results: Subject-Verb Links are Backwards}
\begin{itemize}
      %\item Different dependency formalisms disagree on whether a clause is headed by the aux verb or the main verb.
  \uncover<1->{\item The Link Grammar seems to be inconsistent about whether the auxiliary verb or the main verb is the head of a clause. }
  \uncover<2->{\item Sometimes the governing verb links to the auxilliary, and sometimes to the main, depending on the type of clause. }
  \uncover<3->{\item But the governing verb usually links to the subject when there is one.}
  \uncover<4->{\item So this makes the subject a consistent choice to make the head of a clause.}
\end{itemize}

%governing verb -\> subject -\> auxilliary verb -\> main

\uncover<5->{
\begin{center}
  To fix this, we could edit the link grammar, link parses, or the ILP. %\\ Our technique is only as useful as the input given.
\end{center}
}
    %Question: Which do we pick?
    %Answer: Neither! We actually pick the subject as the head of the clause! 
    %Question: Why??
    %Answer:
    %    The LG folks seem to be inconsistent within their grammar about whether the aux verb or the main verb is the head of a clause. 
    %        Sometimes the governing verb links to aux, and sometimes to main, depending on the type of clause.
    %    But the governing verb usually links to the subject when there is one!
    %    Therefore, the subject ends up being a consistent choice for the head of the clause.  We get governing_verb --> subject --> aux --> main as our annotation scheme.
    %Question: Couldn't this be fixed?
    %Answer: Of course it could.  You could edit the link grammar or the link parses or the ILP.  Our technique is only as useful as the input you give it ...  
}


\section{Conclusions}
\frame{\frametitle{Conclusions}
  \begin{itemize}
  \item Link Grammar parses can be oriented into connected DAGs
  \item A new corpus available for building multi-headed dependency parsers
  \item ILP can be used to help annotate incomplete or missing data in corpora. 
  \end{itemize}
}


\frame{\frametitle{}
  \begin{itemize}
  \item[] Questions?
  \end{itemize}
}



%\nocite{*}
%\bibliographystyle{plain}
%\bibliography{hong+eisner.TLT13.slides.bib}
%\appendix

\end{document}
