%
% File acl2014.tex
%
% Contact: koller@ling.uni-potsdam.de, yusuke@nii.ac.jp
%%
%% Based on the style files for ACL-2013, which were, in turn,
%% Based on the style files for ACL-2012, which were, in turn,
%% based on the style files for ACL-2011, which were, in turn, 
%% based on the style files for ACL-2010, which were, in turn, 
%% based on the style files for ACL-IJCNLP-2009, which were, in turn,
%% based on the style files for EACL-2009 and IJCNLP-2008...
%% Based on the style files for EACL 2006 by 
%%e.agirre@ehu.es or Sergi.Balari@uab.es
%% and that of ACL 08 by Joakim Nivre and Noah Smith

\documentclass[11pt]{article}
\usepackage{LinksToDAG}
\usepackage{times}
\usepackage{url}
\usepackage{latexsym}

\usepackage[usenames,dvipsnames,svgnames,table]{xcolor}
\usepackage{soul}
\newcommand{\Note}[1]{}
\renewcommand{\Note}[1]{\hl{[#1]}}  % comment out this definition to suppress all Notes                                                                                
\newcommand{\TODO}[1]{\Note{TODO: #1}}
\newcommand{\NoteSigned}[3]{{\sethlcolor{#2}\Note{#1: #3}}}
\newcommand{\NoteJE}[1]{\NoteSigned{JE}{LightBlue}{#1}}
\newcommand{\NoteJH}[1]{\NoteSigned{JH}{YellowGreen}{#1}}


\title{Link Parses to Directed Acyclic Graphs}

\author{Juneki Hong and Jason Eisner\\
  Department of Computer Science \\
  Johns Hopkins University \\
  Baltimore, MD 21218, USA \\ 
  {\tt \{juneki,jason\}@cs.jhu.edu} \\
}

\date{}

\begin{document}
\maketitle

\begin{abstract}

We will show that any legal link grammar parse is a connected DAG, with all of the nodes reachable from the root. Every undirected edge in a link grammar parse can be assigned directionality, by specifying the task as an ILP problem given the constraints of acyclicity, connectedness, and the link grammar rules described from a dictionary.

\end{abstract}


\section{Introduction}

\TODO 
% Why would someone be interested in reading this paper?


This can be solved as an Integer Linear program. Encoded in the Zimpl little language \cite{Koch2004}. 


\section{Previous Work}

\TODO



\section{Link Grammars}

Link Grammars are a formalism that describes the links, or relationships between the constituents \cite{SleatorTemperly91}. They are represented as 




\section{ILP Encoding}
We ignored the link parses that the link parser could not find suitable attachments and returned a disconnected graph. 

\section{Conll-Link analysis}







\bibliographystyle{LinksToDAG}
\bibliography{LinksToDAG}

\end{document}
